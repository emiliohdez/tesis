\chapter*{Resumen}
%spacing{1.5}
\noindent Existen un sin número de problemas sociales que afectan la calidad de vida de las personas que conviven en una determinada población, como el ascendente nivel de contaminación, el desempleo, el vandalismo y el escaso transporte urbano, este último muy frecuente en poblaciones pequeñas de México y foco de atención de este escrito. Al haber muy pocas unidades de transporte urbano en una población, el mínimo desequilibrio en los tiempos de llegada de cada autobús a cierta parada oficial trae a los usuarios de este servicio pérdidas de tiempo y muchas veces consecuencias peores derivadas de la impuntualidad provocada, como descuentos en el sueldo e incluso pérdida de trabajo.

Lo que se menciona en las líneas anteriores es la problemática a la  que se buscó dar solución durante el desarrollo del proyecto del que esta tesis trata, un sistema de software basado en tecnologías web y móviles que ayude a los usuarios del sistema de transporte público a evitar tener que realizar largas esperas por el autobús que requieren para llegar a su destino.

Cabe mencionar que el presente escrito habla acerca del desarrollo de uno de las dos módulos que conforman el sistema completo, el segundo para ser exactos. El primer módulo trata acerca del desarrollo de la aplicación móvil que se encarga del rastreo de la unidad del autobús y que envía esa información a un servidor (en lo siguiente llamado \textit{servidor de recolección}), en donde cierta parte de la información recibida será empleada para fines de investigación y la otra parte de la información es la que pasa al servidor web al que se conectan los usuarios móviles y web para la visualización de datos.

El \textbf{capítulo 1} introduce al lector de manera breve al problema que se está abordando describiendolo a detalle, mencionando algunos de sus antecedentes y exponiendo los objetivos que se pretenden alcanzar y las hipótesis ideadas a partir del análisis del problema en cuestión.

El \textbf{capítulo 2} se presentan las bases teóricas que fundamentan este trabajo y que son necesarias para la completa comprensión de este por parte del lector. Entre los conceptos tratados se tienen en su mayoría términos técnicos relacionados a los Lenguajes de programación, tecnologías y formatos utilizados en el desarrollo del software y algunos otros que tratan acerca de las bases geográficas del Sistema GPS utilizado para el rastreo de las unidades de transporte público.

Por su parte, en el \textbf{capítulo 3} es en donde se entra de lleno a la descripción técnica del sistema desarrollado comenzando con una breve explicación sobre la parte o módulo de recolección y procesamiento de datos desarrollada por el Ingeniero en Sistemas Computacionales Leonardo Alvarez Rivera, con el fin de dar a conocer el punto de partida para el desarrollo de esta tesis, posteriormente se continua con la implementación  del software que se ejecuta en el servidor web, describiendo la arquitectura y los patrones de diseño de software que fueron empleados, así como las tecnologías para conexión y transmisión de datos que fueron implementadas.

Continuando con la descripción del desarrollo, el \textbf{capítulo 4} trata sobre la implementación de software realizada del lado del cliente en su versión móvil, haciendo incapié en las interfacez de usuario y su funcionamiento, a demás de la manera en la que se procesan los datos provenientes del servidor.

De manera similar al  capítulo 4, en el \textbf{capítulo 5} se expone la implementación de software realizada en el lado del cliente, pero esta vez en su versión web, explicando de igual manera el funcionamiento de las interfacez y el rol que cada tecnología tuvo en la construcción de las mismas.

Posterior a esto, se presentan las conclusiones obtenidas del desarrollo de este proyecto y el resultado de algunas pruebas realizadas, como el caso de las pruebas de estres y la prueba de la ejecución del software en dispositivos con diferentes especificaciones técnicas, también se incluyen las actualizaciones futuras a realizar y la participación que este sistema tendrá con otros trabajos futuros.

Finalmente se incluyen las referencias a los artículos, trabajos, libros y reportes o aportaciones técnicas que sirvieron de base para el desarrollo del sistema de software.