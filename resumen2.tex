\chapter*{Resumen}
\spacing{1.5}
\noindent Los sistemas de monitoreo o rastreo de objetos comenzaron a ser idealizados a partir de la liberación de la tecnología GPS\footnote{Por sus siglas en inglés \textit{Global Positioning System}.} por parte de la Fuerza Aérea de los Estados Unidos, teniendo como objetivo general el ubicar cualquier tipo de objeto o muchos de ellos sobre la superficie terrestre. Con el tiempo y los avances tecnológicos la idea básica de un sistema de rastreo ha sido adaptada para su implementación en una gran variedad de situaciónes y temas, tales como los relacionados a la seguridad, detección de problemas sociales, cuestiones de logística e inclusive actividades recreativas.

En la presente tesis se describe la implementación de un sistema de software para el monitoreo del transporte urbano con el que se pretende, principalmente, ayudar a los usuarios de este servicio en zonas en las que presenta deficiencias, como el caso de ciudades en las que el servicio cuenta con pocas unidades de transporte, o pequeñas colonias alejadas en las que el transporte puede tardar mucho en llegar.

Este trabajo de desarrollo de software se divide en 5 capítulos, descritos a continuación:

En el \textbf{capítulo 1} se explica... (Marco Metodológico)

En el \textbf{capítulo 2} se aborda... (Marco Teórico)

En el \textbf{capítulo 3} se entra de lleno en ... (Lado del servidor)

En el \textbf{capítulo 4} se describe lo relacionado a... (Lado del Cliente Web)

En el \textbf{capítulo 5} se explica a detalle el desarrollo del (Lado del cliente móvil)

Finalmente se presenta evidencia del sistema trabajando y se plantean avances a futuro... (Kind of this)

