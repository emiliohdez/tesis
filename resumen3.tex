\chapter*{Resumen}
\spacing{1.5}
\noindent A medida que la tecnología avanza, la calidad de vida de las personas y el ambiente en que se desenvuelven se ven mejorados, desde la manera en que se comunican e interactúan hasta el proceso mediante el cual aprenden. Uno de los cambios más notables en los últimos años ha sido el desmesurado aumento en el uso de smartphones\footnote{Traducción al inglés de \textit{Teléfono Inteligente}.} como una herramienta indispensable para el desarrollo de cualquier actividad del día a día, dígase comunicarse con otros a través de llamadas, videoconferencias o mensajes de texto, ejercitarse, tomar fotografías y videos, medir el ritmo cardíaco, ubicar a otros en un mapa, etc.

El presente trabajo de tesis describe parte de un proyecto de software que se está desarrollando en el Centro de Investigación en Matemáticas, A. C. (CIMAT), cuyo objetivo en conjunto es aportar una actualización o mejora al sistema de transporte urbano basada en el uso de dispositivos móviles y tecnologías o estándares web. En la actualidad un alto porcentaje de personas cuentan con un teléfono inteligente o una computadora personal, siendo usuarios del internet y de los sistemas de comunicación digital \cite{INEGI_2017_INTERNET}. Dicho objetivo pretende ser alcanzado a través de la implementación de un sistema de software para monitorear el servicio de transporte urbano de cualquier ciudad o población, en especial aquellas en las que el servicio presenta deficiencias asociadas a la mala administración de las unidades que circulan las rutas de autobuses, o de la incapacidad para cubrir los requerimientos de unidades.

En particular, esta tesis trata acerca de dos componentes que se encargan de la interacción con el usuario final, una aplicación para dispositivos móviles y una aplicación para la web que comparten funciones muy similares. Ambas aplicaciones se encargan de desplegar información de interés para las personas que utilizan el servicio de transporte, como las rutas de la ciudad de manera gráfica, la distribución de los parabuses oficiales para cierta ruta, la ubicación en tiempo real del grupo de autobuses que la circulan en una ruta y de estimaciones de distancias y tiempos de llegada.

Esta tesis se divide en 4 capítulos, que son descritos a continuación:
%
%Es muy importante aclarar que este escrito solo involucra la descripción a detalle de dichas aplicaciones, por lo que la explicación acerca del resto de los componentes del sistema, su arquitectura y el modelo de comunicación entre procesos, quedan fuera del alcance de esta tesis. 
%
%El trabajo de desarrollo de software se presenta dividido en 5 capítulos, descritos a continuación:

En el \textbf{capítulo 1} se introduce al lector de manera breve al problema que se está abordando, describiéndolo de manera concisa, mencionando algunos de sus antecedentes, exponiendo los objetivos que se pretenden alcanzar y planteando las hipótesis ideadas a partir del análisis del problema en cuestión.

El \textbf{capítulo 2} presenta las bases teóricas que fundamentan este trabajo y que son necesarias para la comprensión de éste por parte del lector. Entre los conceptos tratados se tienen en su mayoría términos técnicos relacionados a los lenguajes de programación, tecnologías y formatos utilizados en el desarrollo del software y algunos otros que tratan acerca de las bases geográficas del Sistema de Posicionamiento Global (GPS) utilizado para el rastreo de las unidades de transporte público.

En el \textbf{capítulo 3} se expone una descripción a grandes rasgos del sistema, su fundamentación en los requisitos y necesidades expuestos por parte de los usuarios, los módulos arquitectónicos, el diseño que se propuso con base en los requerimientos y el modelo de datos básico sobre el que fue implementado.

El \textbf{capítulo 4} describe a detalle el desarrollo de las aplicaciones móvil y web que fungen como capa de interacción con el usuario final. Expone información como los requerimientos que satisfacen de primera mano, la arquitectura y modelo empleados para su desarrollo, los patrones de diseño utilizados y el esquema lógico de flujo de datos sobre el que se basan. Finalmente se muestran algunas pruebas de implementación de ambas versiones en diferentes escenarios de interacción.
