
\chapter*{Conclusiones}
\addcontentsline{toc}{chapter}{Conclusiones} 
\noindent Para finalizar este escrito es importante mencionar que la fase de pruebas en ambiente de producción se está realizando actualmente en colaboración con la Universidad de Guanajuato, empleando los rastreadores en las unidades de transporte institucional de la misma y desplegando la información de estas a sus alumnos en ciertos horarios del día. Con esto se pretende mejorar un poco la calidad de vida de los alumnos, principalmente, al darles una herramienta extra que les ayude a evitar tener que utilizar transporte propio o un servicio privado (Taxi o UBER) al desplazarse de un campus universitario a otro.

Con base en lo antes mencionado, es viable concluir que la hipótesis y los objetivos planteados en un inicio fueron cubiertos. Definitivamente la razón principal por la que las personas no optan por utilizar el servicio de transporte público es la incertidumbre en tiempo que éste conlleva, la cual es aminorada al tener información en tiempo real del estado del sistema de transporte.

También es posible expresar que, puesto que el sistema de rastreo en cuestión fue desarrollado siguiendo el modelo más abstracto posible, representa una posibilidad muy grande de automatización en uno de los servicios más utilizados por la población mexicana, y la base para el mejoramiento del servicio al grado de países de primer nivel como Japón, Estados Unidos, Alemania y Canadá, pero con una diferencia muy importante en cuanto a la economización de recursos y presupuestos, y a la robustez y escalabilidad del proyecto.

Cabe recalcar que se prevé que esta integración con las tecnologías de la información puede también traer mejoras en otras problemáticas sociales derivadas del uso masivo de autos particulares, como el tráfico ya definido por las \textit{horas pico} y la excesiva emisión de gases contaminantes por uso de combustibles fósiles.

El desarrollo de este sistema continua en manos de un equipo de desarrolladores de software en el CIMAT, puesto que se busca aportar otras utilidades a los usuarios, como el caso de una extensión que permita planear las rutas de autobuses a utilizar para llegar de un lugar a otro, implementando algoritmos para encontrar el camino más corto en el grafo, tomando en cuenta que el usuario puede necesitar tomar dos rutas de autobuses no consecutivas.

Del mismo modo, se está contemplando la posibilidad de agregar la función de indicar al usuario si la siguiente unidad de transporte en pasar por el parabús de su preferencia se encuentra a su máxima capacidad de pasajeros, utilizando técnicas de visión por computadora y \textit{crowdsourcing}\footnote{Técnica para la obtención masiva de datos provenientes de un numeroso grupo de individuos.}.